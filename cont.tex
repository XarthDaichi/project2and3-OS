\documentclass[11pt]{proc}
\usepackage[spanish]{babel}
\usepackage[utf8]{inputenc}
\usepackage{fancyhdr}
\usepackage{xcolor}
\usepackage{blindtext}


\pagestyle{fancy}
\fancyhf{DOP}
\rhead{Unidad Desconcentrada IC-Alajuela}
\lhead{Programa de tutoría estudiantil}
\rfoot{TEC}
\lfoot{II-2022}

\newcommand{\nombreCurso}{Introducción a la programación }
\newcommand{\noContrato}{examen comprensivo de toda la materia en semana 18 }


\newcommand{\nombreEstudiante}{ ESCRIBA SU NOMBRE COMPLETO AQUÍ }
\newcommand{\numCarne}{ ESCRIBA SU CARNÉ AQUÍ }
\newcommand{\periodo}{ primer semestre 2023}

\title{\textbf{Contrato de aplicación de tutoría de asistencia obligatoria}}
\date{\today}
\begin{document}

\maketitle
\section*{Antecedentes}

El curso de \nombreCurso ha tenido una tasa de reprobación de 50\% a nivel histórico en todo el país. Este contrato pretende ofrecer la tutoría de asistencia obligatoria no solo a estudiantes repitentes, sino también a estudiantes que lleven el curso por primera vez, para que los mismos puedan tener acceso a este tipo de apoyos de manera preventiva.  
Por lo anterior, se ofrecerá a los estudiantes matriculados en los cursos regulares de \nombreCurso, durante el \periodo, la oportunidad de optar por un sistema de evaluación diferenciada que incluye la participación en programa de tutoría de \textbf{asistencia obligatoria}, según los objetivos y especificaciones que se detallan a continuación. 

\section*{Objetivos}
\begin{itemize}
\item Ofrecer espacios de discusión de la materia organizados por estudiantes
\item Generar espacios académicos de estudio
\item Permitir compartir experiencia de estudiantes calificados y avanzados de la carrera con quienes están iniciando.
\item Mejorar la comprensión de tópicos del curso
\item Mejorar los niveles de promoción mostrados en los cursos.
\item Favorecer el desarrollo de estrategias para aprender a aprender.
\item Estimular el aprendizaje cooperativo. 
\item Fortalecer en los estudiantes hábitos y estrategias de estudio.
\end{itemize}

\section*{Evaluación}
En caso de aceptarse este contrato, la evaluación indicada en la carta al estudiante pasaría a tener un valor del 90\% del total del curso, mientras que el 10\% restante sería el reportado por los tutores de acuerdo con la asistencia y el cumplimiento de tareas. Se sugiere aportar en cada tarea una discusión del proceso creativo para llegar a cada solución, en lugar de sólo la solución per sé. 

\section*{Procedimiento}
\begin{enumerate}
\item Se realizarán 8 prácticas extra clase que los estudiantes deberán resolver de manera individual previo a la tutoría. Para la revisión de dichas prácticas los estudiantes deberán asistir a 16 sesiones de tutoría de acuerdo con el esquema que se plantee.
\item Las prácticas se desarrollarán con dos apartados: \subitem Resolución de problemas \subitem Análisis de errores. 
\item La primera sección contempla el desarrollo de la propuesta de solución de problemas para cada ejercicio. La segunda parte, el análisis de errores implica una recapitulación de los fallos cometidos frente a los ejercicios; posterior al espacio de reflexión de las tutorías, con el objetivo de desarrollar habilidades de auto monitoreo y control del propio aprendizaje. 

\item Para 8 tutorías el estudiante deberá llevar resuelta una práctica, que será asignda en la tutoría anterior (en que no había práctica). Si no entrega la tarea antes de la hora indicada, previo a la tutoría, su asistencia no será contada en la misma. No es requisito que la práctica esté resuelta correctamente para poder participar en la tutoría, el requisito obligatorio es la entrega de todas las soluciones. Si algún ejercicio no se pudiera completar, deben presentarse al menos 5 intentos diferentes para lograr la solución o una explicación del análisis de la solución.
\item La nota de tutoría será el promedio de asistenacias a tutorías. (En aquellas en las que deben presentar prácticas, deben haberse presentado todas las soluciones a los ejericicios).
\item Si el estudiante NO asiste a la sesión de tutoría, no se asignarán puntos por la práctica resuelta, aún y cuando la haya completado, salvo justificación médica y/o psicológica emitida por la CCSS, el Centro de Salud del TEC o el Departamento de Orientación y Psicología.
\item \textbf{Cláusula de rescisión}: Si el estudiante considera que la evaluación especificada en este contrato no le favorece, podrá presentar su solicitud de rescisión por escrito a más tardar en la \textbf{quinta} semana lectiva. De rescindir el contrato, regirá para el estudiante los criterios de evaluación definidos en la carta al estudiante del curso y se regulará según lo estipulado en la carta al estudiante.
\end{enumerate}

\section*{Detalles importantes}
Las prácticas serán resueltas de manera individual, con anterioridad a la sesión de tutoría.
Se realizarán 8 prácticas distribuidas a lo largo del semestre. Las mismas serán asignadas con al menos cinco días de anticipación y deberán presentarse resueltas en forma individual en la fecha indicada, previo a la tutoría como requisito para participar en la misma (la entrega de la misma se registrará en la tutoría).
Cabe aclarar que las prácticas se evaluarán de forma individual. No se calificará el resultado, sino el grado de análisis y profundidad de la misma. Sin embargo, tampoco se asignará puntuación a trabajos realizados en forma superficial. En caso de tener dificultades con ejercicios, se podrá solicitar asesoría de tutores regulares o docentes del curso. No se recibirán prácticas cuando los estudiantes no hayan participado en las sesiones de análisis generadas en la tutoría.

\section*{Asuntos administrativos}
Para gozar de este beneficio los (as) estudiantes estarán en la obligación de cumplir con lo siguiente:
\begin{enumerate}
\item Cada estudiante interesado debe estar matriculado en algún grupo de \nombreCurso para poder participar en esta modalidad de curso.
\item Asistir semanalmente a las sesiones de tutoría diseñadas específicamente para atender las necesidades de esta población.
\item Asistir al menos al 80\% de las lecciones del curso (asistencia obligatoria).
\item El estudiante que habiendo participado en este tipo de evaluación repruebe el curso, podrá volver a participar de esta modalidad de evaluación sólo en aquellos casos que haya cumplido con la entrega de al menos 80\% de las prácticas y las tutorías. 
\item La participación en esta modalidad de evaluación es voluntaria, pero una vez que el estudiante haya manifestado su consentimiento para aceptar las nuevas condiciones de evaluación y luego de vencido el periodo estipulado por la cláusula de retiro, no será posible regresar al sistema de evaluación anterior definido en la carta al estudiante del curso.
\item El estudiante queda protegido por la ley 8968, en relación la protección de la persona frente al tratamiento de sus datos personales. 
\end{enumerate}

\bigskip

Yo \nombreEstudiante, estudiante de la carrera Ingeniería en Computación, carné \numCarne , no ( ) sí ( ) acepto las condiciones estipuladas en este contrato para que se me aplique esta propuesta de evaluación con sus beneficios y obligaciones.    \\
Fecha: \today  \\Firma: 
\bigskip
\bigskip
\bigskip
\bigskip

Docente del curso\\
Nombre: Eddy Ramírez Jiménez 
\\Firma y sello de la Unidad: 



\end{document}